\documentclass{article}

\usepackage{arxiv}


\usepackage[utf8]{inputenc} % allow utf-8 input
\usepackage[T1]{fontenc}    % use 8-bit T1 fonts
\usepackage{hyperref}       % hyperlinks
\usepackage{url}            % simple URL typesetting
\usepackage{booktabs}       % professional-quality tables
\usepackage{amsfonts}       % blackboard math symbols
\usepackage{nicefrac}       % compact symbols for 1/2, etc.
\usepackage{microtype}      % microtypography
\usepackage{lipsum}         % Can be removed after putting your text content
\usepackage{graphicx}
\usepackage{natbib}

% Our files
\usepackage{packages/colab}
\usepackage{packages/math}

% Packages that should be loaded after amsmath
\usepackage{cleveref}       % smart cross-referencing

\title{A template for the \emph{arxiv} style}

% Here you can change the date presented in the paper title
%\date{September 9, 1985}
% Or remove it
%\date{}

%\author{ Dhruvesh Patel\thanks{Use footnote for providing further
%		information about author (webpage, alternative
%		address)---\emph{not} for acknowledging funding agencies.} \\
%	College of Information and Computer Sciences\\
%	University of Massachusetts Amherst\\
%	\texttt{dhruveshpate@umass.edu} \\
%	%% examples of more authors
%	\And
%	 Dhruvesh Patel\thanks{Use footnote for providing further
%		information about author (webpage, alternative
%		address)---\emph{not} for acknowledging funding agencies.} \\
%	College of Information and Computer Sciences\\
%	University of Massachusetts Amherst\\
%	\texttt{dhruveshpate@umass.edu} \\

%}

% Uncomment to override  the `A preprint' in the header
%\renewcommand{\headeright}{Technical Report}
%\renewcommand{\undertitle}{Technical Report}
\renewcommand{\shorttitle}{\textit{arXiv} Template}

%%% Add PDF metadata to help others organize their library
%%% Once the PDF is generated, you can check the metadata with
%%% $ pdfinfo template.pdf
\hypersetup{
pdftitle={A template for the arxiv style},
pdfsubject={q-bio.NC, q-bio.QM},
pdfauthor={Dhruvesh Patel, Patel Dhruvesh},
pdfkeywords={First keyword, Second keyword, More},
}

\begin{document}
\maketitle
\begin{abstract}
	\lipsum[1]
\end{abstract}
\section{Introduction}
\label{sec:intorduction}

You can add comments. \dhruvesh{To add comments, add a command in the colab.sty.}
\section{Related Work}\label{sec:related work}
\section{Method}\label{sec:method}

\subsection{Background}\section{sec:background} % Always use "sec:" as the prefix in labels regardless of whether it is a section, subsection or subsubsection.

% keywords can be removed
%\keywords{First keyword \and Second keyword \and More}


Here is an example usage of the two main commands (\verb+citet+ and \verb+citep+): Some people thought a thing \citep{kour2014real, hadash2018estimate} but other people thought something else \citep{kour2014fast}. Many people have speculated that if we knew exactly why \citet{kour2014fast} thought this\dots


\bibliographystyle{unsrtnat}
\bibliography{references}  %%% Uncomment this line and comment out the ``thebibliography'' section below to use the external .bib file (using bibtex) .


\end{document}
